\documentclass[aspectratio=169,10pt,english]{beamer} 
\usetheme{cgs}

\usepackage{babel}
%\usepackage[latin1]{inputenc}
%\usepackage[T1]{fontenc}
\usepackage{graphicx}
\usepackage{amsmath}
\usepackage[labelformat=empty,textfont=small,justification=centering]{caption}
\usepackage[labelformat=empty,textfont=scriptsize,justification=centering]{subcaption}
\usepackage{datetime}
\usepackage{textpos}
\usepackage{tikz}
\usepackage{array}
\usepackage{multirow}
%\usepackage{gitinfo}

\usepackage{inconsolata}
\usepackage{listings}
\input{glsl-lstset}


\hypersetup{bookmarksopen=false}

\providecommand*{\what}[1]{\widehat{#1}}
\providecommand*{\wtilde}[1]{\widetilde{#1}}

\title{Grafikprogrammierung mit C++ und OpenGL}
\author{Stefan Buschmann, Daniel Limberger, Amir Semmo}
\institute{Hasso-Plattner-Institut}
\date{SoSe~2013}

\setbeamercolor{postit}{fg=black,bg=white}

\newcommand<>{\fullsizegraphic}[2]
{
	\begin{tikzpicture}[remember picture,overlay]
	\node[at=(current page.center)]
	{
		\includegraphics[width=\paperwidth]{#1}
	};
	\end{tikzpicture}

	\vspace{-0.33\paperheight}\hfill
	\begin{beamercolorbox}[wd=0.5\paperwidth,ht=11mm,dp=1ex,right,rightskip=1cm]{postit}
		\fontsize{8mm}{0}\selectfont  \uppercase{\textbf{#2}}
		\vspace{2mm}
	\end{beamercolorbox}
	\hspace{-2cm}
}

\newcommand<>{\grayout}{\textcolor{gray}}