%!TEX root = foo-thesis.tex

\chapter{Evaluierung}

\section{Trainings- und Testdaten}

% \section{Experimente}

\section{Preprocessing} 

\section{Ansatz Feature-Extraction} 

\subsection{Features}

\subsection{Scales}

\subsection{Ein- und mehrwertige Features} % hier darauf eingehen, dass durch Histogramme größere Scales erspart bleiben

\subsection{Uniqueness}

\subsubsection{Sonstige Verarbeitungsschritte} 

\subsubsection{Prediction per Random Forest} 

\section{Ansatz Deep Learning}

\section{Postprocessing}

% immer subsections mit (tlw. Vergleichen mittels) Bildern, Plots, Metrics und Performance PLUS Erklärungen/Vermutungen
% auch eingehen auf Speicherverbrauch (nicht nur Dauer) und Tradeoff dazwischen -> evtl. auch kleine Techniken zur Speicherreduktion (mehrere 0.0er Bins in Folge, ist auch weniger Noise)

% Kapitel noch anders strukturieren, mehr auf Vergleich aufbauen zwischen beiden Ansätzen
% PointNet mit gleichen Scales wie ich (einzelne ausprobieren und vergleichen mit Gesamtleistung von mir)
% PointNet: schauen, ob Parameter wie neighborhood size anzupassen sind

% zum Schluss nochmal alles probieren mit zusätzlichem Gehweg, Bordsteinkante plus Noise; je nach Ergebnissen darauf eingehen oder nicht...
% darauf eingehen, dass künstliche Schlaglöcher erzeugt?