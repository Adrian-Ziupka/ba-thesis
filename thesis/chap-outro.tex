%!TEX root = foo-thesis.tex

\chapter{Fazit und Ausblick}
\label{chap:outro}

Zusammenfassend lässt sich sagen, dass der gesamte Ablauf, wie er sich im Voraus erdacht wurde, zumindest grundlegend bewährt hat. Dies beginnt mit der im anderen Projektteil behandelten optimierten Bodenerkennung, welche die Basis bildet für die folgende Straßenextraktion. Erst mit Ermittlung der Straße kann die Erkennung von Straßenschäden beginnen. Die Nutzung mehrerer Scales hat sich als nützlich erwiesen, um mit unterschiedlichen Granularitäten der teils sehr unregelmäßigen Objekte umzugehen. Der Einsatz mehrwertiger Features in Form von Histogrammen sorgt zwar für einen erhöhten Speicherbedarf, kann aber gleichzeitig neben einer höheren Aussagekraft zum Teil dafür sorgen, rechenaufwändige höhere Scales zu vermeiden und die Laufzeit somit niedriger zu halten. Falls größere Scales etwa wegen umfangreicherer Objekte doch nötig werden, kann eine flexible Dichtereduktion der Punktwolke erfolgen und die gesuchten Features werden nach Distanz gewichtet approximiert. Die genutzten Features auf Basis der Intensitäten und Formen der lokalen Nachbarschaft erreichen, dass zumindest die eindeutigen Punkte einer Klasse - etwa Flickstellen, die sich merklich abheben - auch korrekt klassifiziert werden. Für die restlichen Punkte bedarf es wohl weiterer Features, die für solche Fälle eine größere Aussagekraft besitzen. Das implementierte \textit{Uniqueness}-Konzept dient bisher vornehmlich der Datenreduktion für den Transfer der Featurevektoren, ist aber direkt einsetzbar für weitere Anwendungsfälle, die von einer sinnvollen Punktereduktion profitieren. Die Vorhersage der Klassen über \textit{Random Forests} sorgt für schnelle und - in Kombination mit den Histogrammen - robuste Ergebnisse. Auch \texttt{PointNet} hat gezeigt, obwohl dessen \textit{Predictions} viel Rauschen und kaum gefundene Flickstellen beinhalten, dass es gewinnbringend für die Erkennung von Straßenschäden eingesetzt werden und auffällige Punkte ermitteln kann. Mittels weiterer und vielfältiger Trainingsdaten sowie dem Optimierungssystem der Webplattform für \textit{Deep-Learning}-Modelle könnte diese Architektur künftig die Basis der Straßenzustandserkennung bilden. \\\\
Für den \textit{Feature-Extraction}-Ansatz bieten sich verschiedene Verbesserungen und Erweiterungen an. Zum einen könnte die Integration in ein einziges abgeschlossenes System vollzogen werden, vorzugsweise das \texttt{PCTool}, um den zusätzlichen Mehraufwand von Datentransfers zu vermeiden. Für die Straßenextraktion, einer notwendigen Vorstufe für die Schadenserkennung, könnten elaboriertere Verfahren entwickelt werden, die den Anteil der reinen Straße gegenüber Gehweg und sonstigen Oberflächen weiter erhöhen. Diese können neben der bisherigen Punktdichtebetrachtung etwa auch auf der Erkennung von Bordsteinkanten als seitliche Grenzen basieren. Ebenfalls sind hierfür \textit{Deep-Learning}-Ansätze denkbar. \\
Ein großer Fokus könnte auf der Nutzung von noch aussagekräftigeren Features liegen, welche die Basis aller \textit{Predictions} darstellen. Das können sowohl Variationen der bisher genutzten Features sein - darunter fallen auch nicht-lineare Histogramme, welche die vorkommenden Werte besser abbilden können - als auch komplett andere Maße. Globale Features mit der gesamten Punktwolke als Ausgangslage können ebenfalls genutzt werden und Informationen in die Featurevektoren bringen, die durch die lokale Nachbarschaft allein nicht auszudrücken sind. \\
Insbesondere für die Berechnung der \textit{uniquen} Punkte dürfte eine initiale Ermittlung des groben Straßenzustands hilfreich sein. Dies könnte durch einen weiteren Fingerabdruck der Straßenpunktwolke repräsentiert werden und beispielsweise ausdrücken, ob es sich eher um eine Straße in der Innenstadt in sehr gutem Zustand handelt oder um eine marode Landstraße. Diese Daten würden schließlich genutzt werden, um Parameter des \textit{Uniqueness}-Konzepts sowie Ober- und Untergrenzen der Histogramme besser abzuschätzen und letztlich sowohl die ermittelte \textit{Uniqueness} als auch die \textit{Predictions} für die konkrete Punktwolke zu verbessern. Mit einer so verfeinerten \textit{Uniqueness} könnte außerdem die Geschwindigkeit und Genauigkeit einer Registrierung gesteigert werden, die vor allem auf den beständigen Gullys basiert. In Verbindung damit ist auch eine Dokumentation der Zustandsentwicklung einer Straße denkbar: Anhand der Änderungen der \textit{uniquen} Punkte, zuvorderst der Schadensklassen, über einen gewissen Zeitraum könnten Problemstellen ausgemacht und ggf. prognostiziert werden. \\
Eine Fortführung der Schadenserkennung mittels \textit{Deep-Learning}-Ansätzen kommt auch in Betracht. Dabei könnten zum einen noch mehr Konfigurationen von \texttt{PointNet} auf ihre Qualität getestet werden. Zum anderen wären auch gänzlich andere Architekturen interessant, die auf direkter Punktbasis arbeiten, wozu der Nachfolger von \texttt{PointNet} zählt. Allen Ansätzen gemein ist jedoch ein ausgereifteres \textit{Postprocessing}, das seinen Fokus je nach Anwendungsfall anders legen könnte. Falls zum Beispiel Flickstellen besser und vollständiger erkannt werden sollen, können die einzelnen erkannten \textit{Cluster} zu approximierten Rechtecken verbunden werden. Wenn hingegen vor allem die Detektion der Gullys relevant ist, kann mehr Aufwand betrieben werden, die Umrisse und das planare Innere präzise zu erkennen. \\
Wie generell für Ansätze des maschinellen Lernens wären auch hier mehr und vor allem unterschiedlichere Trainingsdaten wichtig, um auf vielfältigen realen Daten gute Ergebnisse zu liefern. Dazu zählen etwa Schlaglöcher mit größeren Tiefen oder länglicheren Umrissen. Auch Gullys mit anderen Größen und weitere Klassen wie Risse oder leichte Erhebungen könnten von Interesse sein. Außerdem könnten dabei auch mehr verschiedene Laserscanner zum Einsatz kommen, wofür das \textit{Preprocessing} der Intensitäten besonders nützlich sein dürfte. Auf diese Weise könnte der bisherige \textit{Proof of Concept} erweitert werden, um die Vielfalt und Komplexität der Realität besser zu verarbeiten. \\
Schließlich könnte mit der Erkennung von Schäden und Unzulänglichkeiten einer Straße auch eine Bewertung ihres Zustands erfolgen. Dabei sind verschiedene Verfahren denkbar, etwa eine einfache Kumulation der Punkte pro Schadensklasse in der gesamten Punktwolke. Solche Schritte könnten auch in kleinerem Maßstab vorgenommen werden, indem etwa gleichmäßige Raster in der Punktwolke erzeugt und einzeln bewertet werden. Je nach notwendiger Genauigkeit der Zustandsbewertung könnte die Klassifikation dabei in feineren Abstufungen geschehen: So würden zum Beispiel Schlaglöcher anhand ihrer Größe und Tiefe eingeteilt in \textit{kleiner}, \textit{mittlerer} oder \textit{großer} Schaden. Eine solche Bewertung kleinerer Abschnitte wäre dann die Basis, um Priorisierungen für Reparaturen zu schaffen und Straßenschäden somit schneller und effizienter zu beheben.