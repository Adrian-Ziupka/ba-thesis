%!TEX root = foo-thesis.tex

\chapter{Fazit und Ausblick}

\mediumtodo{Literaturverzeichnis korrigieren bzgl. Article/Inproceedings, Editors und Roberts Titel ergänzen}
\mediumtodo{Reihenfolge des Literaturverzechnis ordnen}

\mediumtodo{Fazit schreiben (inkl. kurzer Zusammenfassung)}
% hier Fazit \\\\

Für den Feature-Extraction-Ansatz bieten sich verschiedene Verbesserungen und Erweiterungen an. Zum einen könnte die Integration in ein einziges abgeschlossenes System vollzogen werden, vorzugsweise das PCTool, um den zusätzlichen Overhead von Datentransfers zu vermeiden. Für die Straßenextraktion, einer notwendigen Vorstufe für die Schadenserkennung, könnten elaboriertere Verfahren entwickelt werden, die den Anteil der reinen Straße gegenüber Gehweg und Noise weiter erhöhen. Diese können neben der bisherigen Punktdichtebetrachtung etwa auch auf der Erkennung von Bordsteinkanten als seitliche Grenzen basieren. \\
Ein großer Fokus könnte auf der Nutzung von noch aussagekräftigeren Features liegen, die ja die Basis aller Predictions sind. Das können sowohl Variationen der bisher genutzten Features sein - darunter fallen auch nicht-lineare Histogramme, welche die vorkommenden Werte besser abbilden können - als auch komplett andere Maße. Globale Features mit der gesamten Punktwolke als Ausgangslage können ebenfalls eingearbeitet werden und zusätzliche Informationen in die Featurevektoren bringen, die mit lokalen Nachbarschaften allein nicht auszudrücken sind. \\
Insbesondere für die Berechnung der uniquen Punkte dürfte eine initiale Ermittlung des groben Straßenzustands hilfreich sein. Dies könnte durch einen weiteren Fingerabdruck der Straßenpunktwolke repräsentiert werden und beispielsweise ausdrücken, ob es sich eher um eine Straße in der Innenstadt in sehr gutem Zustand handelt oder um eine marode Landstraße. Diese Information würde schließlich genutzt werden, um Parameter des Uniqueness-Konzepts sowie Ober- und Untergrenzen der Histogramme besser abzuschätzen und letztlich sowohl die ermittelte Uniqueness als auch die Predictions für die konkrete Punktwolke zu verbessern. Mit einer so verfeinerten Uniqueness könnte außerdem die Geschwindigkeit und Genauigkeit einer Registration gesteigert werden, die vor allem auf den beständigen Gullys basiert. In Verbindung damit ist auch eine Dokumentation der Zustandsentwicklung einer Straße denkbar: Anhand der Änderungen der uniquen Punkte, zuvorderst der Schadensklassen, über einen gewissen Zeitraum könnten Problemstellen ausgemacht und eventuell sogar prognostiziert werden. \\
Eine Fortführung der Schadenserkennung mittels Deep-Learning-Ansätzen ist ebenfalls denkbar. Dabei könnten zum einen noch mehr Konfigurationen von PointNet auf ihre Qualität getestet werden. Zum anderen wären auch gänzlich andere Architekturen interessant, die auf direkter Punktbasis arbeiten, wozu auch der Nachfolger von PointNet zählt. Allen Ansätzen gemein ist jedoch ein ausgereifteres Postprocessing, das seinen Fokus je nach Anwendungsfall anders legen könnte. Falls zum Beispiel Flickstellen besser und vollständiger erkannt werden sollen, können die einzelnen erkannten Cluster zu annähernd Rechtecken verbunden werden. Wenn hingegen vor allem die Detektion der Gullys relevant ist, kann mehr Aufwand betrieben werden, die Umrisse und das planare Innere präzise zu erkennen. \\
Wie generell für Machine-Learning-basierte Ansätze wären auch hier mehr und vor allem unterschiedlichere Trainingsdaten wichtig, um auf vielfältigen realen Daten gute Ergebnisse zu liefern. Dazu zählen etwa Schlaglöcher mit größeren Tiefen oder länglicheren Umrissen. Auch Gullys mit anderen Größen und weitere Klassen wie Risse oder sonstige Straßenobjekte könnten von Interesse sein. Außerdem würden bei weiteren Daten auch mehr verschiedene Scanner zum Einsatz kommen, wofür sich das Preprocessing der Intensitäten auszahlen sollte. Auf diese Weise könnte der bisherige \textit{Proof of Concept} erweitert werden, um die Vielfalt und Komplexität der Realität besser zu verarbeiten. \\
Schließlich könnte mit der Erkennung von Schäden und Unzulänglichkeiten einer Straße auch eine Bewertung ihres Zustands erfolgen. Dabei sind verschiedene Verfahren denkbar, etwa eine einfache Kumulation der Punkte pro Schadensklasse in der gesamten Punktwolke. Solche Schritte könnten jedoch auch in kleinerem Maßstab vorgenommen werden, indem mit gängigen Geoinformationssystemem gleichmäßige Raster in der Punktwolke erzeugt und einzeln bewertet werden. Je nach notwendiger Genauigkeit der Zustandsbewertung könnte die Klassifikation in feineren Abstufungen geschehen: So würden zum Beispiel Schlaglöcher anhand ihrer Größe und Tiefe eingeteilt werden in \textit{kleiner}, \textit{mittlerer} oder \textit{großer} Schaden. In jedem Fall wäre eine solche Bewertung kleinerer Abschnitte dazu geeignet, einfach Priorisierungen für Reparaturen zu schaffen und Straßenschäden somit schneller und effizienter zu beheben.