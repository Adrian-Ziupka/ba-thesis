%!TEX root = foo-thesis.tex

\chapter{Implementierung}

\section{Systemüberblick} % u.a. mit pctool, pcviewer, PCNN statt extra kurzer Abschnitt zu PCNN/PointNet-Impl

\section{Preprocessing} % Pauls Ground Detection, Street Extraction

\section{Ansatz Feature-Extraction} 

\subsection{Features} % Gewinnung/Berechnung der Features, Intensity contraint und -Collector, EigenvalueCalculator und darüber Curvature & Co., etc.

\subsection{Scales} % einfach über kd-tree

\subsection{Ein- und mehrwertige Features} % Histogram-Klsse etwas erläutern (aber nicht die hässlichen Parts...)

\subsection{Uniqueness} % KL-Divergenz aufschreiben, über Histogramme argumentieren

\subsection{Sonstige Verarbeitungsschritte} % Intensity-Preprocessing; Formel aus FPFH-Paper zur Approximation, zuvor L1-Normalisierung der Distanzen bzw. weights

\subsection{Prediction per Random Forest} % RF aus scikit-learn erklären inkl. wichtiger Parameter

% \section{Ansatz Deep Learning} % nicht selbst implementiert, siehe Systemüberblick

\section{Postprocessing} % einfach auch wieder in kleinem Scale Klassen der Nachbarschaft anschauen