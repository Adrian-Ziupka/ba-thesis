%!TEX root = foo-thesis.tex

\addchap{Zusammenfassung}

\pfinal
3D-Mobile-Mapping-Punktwolken können die Realität in einer Präzision von wenigen Millimetern genau abbilden. Deren unstrukturierte Natur macht es aber herausfordernd, direkt anhand der Punkte Objektklassifizierungen vorzunehmen. Gleichzeitig bieten vermehrt verfügbare Laserscan-Technologie sowie die erhöhte Rechenleistung der letzten Jahre Möglichkeiten, diese großen Datenmengen effizient zu verarbeiten. \\
Ein möglicher Anwendungsfall stellt die automatisierte Erkennung von Straßenschäden in Punktwolken dar, die zuvor mit speziell ausgerüsteten Fahrzeugen aufgenommen wurden. Eine solche semantische Segmentierung, also Einteilung nach Klassen, etwa von Schlaglöchern oder Flickstellen bietet das Potenzial, die bisher meist manuell und damit kostenintensiv durchgeführten Absuchungen nach Schäden durch Straßenbaubehörden zu ergänzen oder abzulösen. Die Analyse auf den 3D-Daten bietet dabei, im Gegensatz etwa zu Bildern der Punktwolke, zusätzliche wertvolle Tiefeninformationen. \\
In dieser Arbeit werden zwei Ansätze zur Detektion von Schäden auf direkter Punktbasis erläutert und miteinander verglichen. Der erste Ansatz basiert auf wohlüberlegten und durch Experimente verfeinerten Features, welche die lokale Nachbarschaft jedes Punktes charakterisieren. Dem zweiten Ansatz liegt das in der Forschung zu semantischer Segmentierung von Punktwolken bekannte neuronale Netz \textit{PointNet} zugrunde, das ähnliche Features versucht automatisch aus den Nachbarschaften zu extrahieren. \\
Dabei zeigte sich, dass % beenden
\mediumtodo{hier Zmsf. des Vergleichs der Ansätze bzgl. Qualität und Performance}

\cleardoublepage