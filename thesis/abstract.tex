%!TEX root = foo-thesis.tex

\addchap{Zusammenfassung}

Die schnelle und umfassende Erkennung von Schäden auf der Fahrbahnoberfläche ist ein wichtiger Aspekt der Straßenpflege und Unfallvorsorge. Gerade Absenkungen wie Schlaglöcher stellen bei höheren Geschwindigkeiten und Tiefen eine erhebliche Gefahr für die Kontrolle des Fahrzeugs und somit ein potenzielles Unfallrisiko dar. Bisher werden solche Absuchungen allerdings meist manuell und kostenintensiv, zum Teil auf Basis von Videoaufnahmen, von Straßenbaubehörden durchgeführt. \\
Laserscans ermöglichen es, die Umgebung digital in Form von 3D-Punktwolken zu repräsentieren, also unstrukturierten Mengen von Punkten im Raum. Aufgrund der steigenden Verfügbarkeit dieser Technologie sowie speziell ausgerüsteten Fahrzeugen, die zu solchen Aufnahmen fähig sind, stehen heute vermehrt Daten von Straßen als 3D-Punktwolken bereit. Die durch dieses sogenannte \textit{Mobile Mapping} erstellten 3D-Punktwolken können die Realität in einer Präzision von wenigen Millimetern genau abbilden, was sehr umfangreiche Datenmengen zur Folge hat. Des Weiteren sind Objekte wie Schlaglöcher und Flickstellen wegen ihrer komplexen, unregelmäßigen Formen nur schwer algorithmisch zu erfassen. Aus diesem Grund werden für die semantische Segmentierung, also die Einteilung der 3D-Punktwolke nach Klassen, Verfahren des maschinellen Lernens genutzt. Diese besitzen das Potenzial, die Erkennung von Straßenschäden zu automatisieren und letztlich die Behebung solcher Schäden durch die Behörden effizienter zu gestalten. \\
In dieser Arbeit werden zwei Ansätze zur Detektion von Straßenschäden erläutert und miteinander verglichen. Beide machen dabei von der dreidimensionalen Struktur der Punktwolken Gebrauch, die zusätzliche Tiefeninformationen bietet. Der erste Ansatz basiert auf manuell festgelegten und durch Experimente bestätigten Features, welche die lokale Nachbarschaft jedes Punktes charakterisieren. Auf diesem Ansatz soll der Fokus der Arbeit liegen. Dem zweiten Ansatz liegt das künstliche neuronale Netz \texttt{PointNet} zugrunde, das versucht, ähnliche Features automatisch aus den Nachbarschaften zu extrahieren. \\
Dabei zeigt sich, dass der Ansatz mit manuellen Features Straßenobjekte wie Gullys und Fahrbahnmarkierungen gut erkennen kann. Auch von Schlaglöchern und Flickstellen werden meist zumindest Teile korrekt klassifiziert, wobei die Flickstellen eher unbeständig erkannt werden. Die Ergebnisse von \texttt{PointNet} zeigen grundsätzlich ähnliche Ergebnisse mit deutlich mehr Rauschen: Gullys und Schlaglöcher werden in ordentlichem Maße gefunden, Flickstellen dagegen kaum.

\cleardoublepage