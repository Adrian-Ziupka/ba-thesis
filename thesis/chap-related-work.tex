%!TEX root = foo-thesis.tex

\chapter{Verwandte Arbeiten}
\label{chap:rel-work}

An dieser Stelle sollen andere Arbeiten angeschnitten werden, die auch Objekterkennung in Punktwolken thematisieren und zum Teil verwandte Zielsetzungen haben.

\section{Objekterkennung in Punktwolken}

Die semantische Segmentierung von Punktwolken, das bedeutet die Einteilung der Punkte in verschiedene Klassen, ist ein in der Forschung intensiv behandeltes Gebiet, welches durch kontinuierliche Steigerung der Rechenkapazitäten zunehmend an Bedeutung gewonnen hat. Da die Punktwolken oftmals durch an Fahrzeugen angebrachten Laserscannern erfasst werden, liegt der Fokus vieler Arbeiten auf der Erkennung von städtischen Objekten, meist mit entsprechender Größe: Fassaden, Bäume, Dächer oder Laternen sind Beispiele dafür. Zu diesem Zweck kommen verschiedene Features pro Punkt zum Einsatz, die großteils einen geometrischen oder reflektivitätsbasierten Hintergrund haben. Darunter fallen die in der Einleitung erwähnten approximierten Normalen oder Krümmungswerte wie \textit{Curvature}, welche die Form der umgebenden Nachbarschaft beschreiben, aber auch Informationen zu der Stärke der Reflektivität eines Punktes \citep{Thomas.etal-2018, Zaboli.etal-2019}. Um die Klassifikation zu erleichtern und mit der teils hohen Varianz zwischen einzelnen Punkten umzugehen, werden in einigen Ansätzen die Punkte zunächst zu größeren Gruppen aggregiert und fortan als Einheit behandelt. In \cite{Vosselman-2013} beispielsweise werden dazu zunächst Segmente gebildet, die planare Flächen zusammenfassen sollen. In \cite{Han.etal-2018} findet sich eine chronologische Auflistung und Erläuterung von Features, die sowohl lokale als auch globale Charakterisierungen von Punktwolken ermöglichen sollen. Moderne Ansätze bauen häufig auf \textit{Deep-Learning}-Modellen auf, welche direkt auf drei- oder mehrdimensionalen Daten arbeiten und erwähnte Features eigenständig extrahieren sollen. Eine bekannte Architektur ist \texttt{PointNet}\cite{Charles.etal-2017}, welche in dieser Arbeit noch genauer vorgestellt und auch getestet wird. Bei \cite{Landrieu.Simonovsky-2018} werden hingegen \textit{Superpoint Graphs} gebildet, welche die Struktur einer Punktwolke und die Beziehungen zwischen Objekten in ihr kompakt repräsentieren sollen und auf denen anschließend ein \textit{Convolutional Network} operieren kann.

\section{Erkennung von Straßenschäden in Punktwolken}

Um mit der unregelmäßigen Struktur von Punktwolken umzugehen, basieren viele Verfahren der Erkennung von Straßenschäden auf bewährten Methoden. Eine Auflistung und Bewertung dieser findet sich bei \cite{Wenming.etal-2020}. Darunter fallen insbesondere Verfahren der Bildsegmentierung oder Kantenerkennungen (\textit{Edge Detection}). Darüber hinaus können auf von Punktwolken gemachten Bildern \textit{Convolutional Neural Networks} eingesetzt werden, die für die Nutzung auf regelmäßigen Bildern spezialisiert sind. Es existieren auch Ansätze, die Straßenschäden direkt in den Punktwolken erkennen wollen. Auf diese wird zum Teil noch mehrmals in der Arbeit verwiesen. Die Schwierigkeit besteht hierbei neben der Unstrukturiertheit von Punktwolken in den standardmäßig ungleichmäßigen Punktdichten. Einige aktuelle Ansätze wie \cite{Famili.etal-2021} und \cite{Gezero.Antunes-2019} setzen auf rein mathematische Herangehensweisen und komplexe geometrische Beschreibungen von Querschnitten der Fahrbahnoberfläche. Mit diesen Methoden sollen etwa Spurrillen ausfindig gemacht werden. Die Arbeiten von \cite{Li.Cheng-2018} und \cite{Zhiqiang.etal-2019} nutzen Verfahren des maschinellen Lernens, nämlich \textit{Random Forests}. Erstere der beiden Arbeiten setzt ihren Fokus nicht auf Straßenschäden, doch kann Bordsteinkanten erkennen und arbeitet darüber hinaus mit sogenannten \textit{Supervoxeln}, welche Punkte einer Nachbarschaft mit vergleichbaren Eigenschaften bündeln. \\\\
Der in dieser Arbeit entwickelte Ansatz baut auf einzelnen Features pro Punkt auf und beschränkt sich auf lokale Features, also solche, die nur eine begrenzte Nachbarschaft betrachten. Außerdem wird jeder Punkt unabhängig behandelt und seine Klasse vorhergesagt, es findet keine vorherige Aggregation von in bestimmten Eigenschaften \textit{ähnlichen} Punkten statt. Das gewählte Modell ist der \textit{Random Forest}, was in Kapitel \ref{chap:prediction} genauer begründet wird.